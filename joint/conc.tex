\section{Summary}

This thesis starts with a general introduction to differential geometry,
tensor calculus and general relativity. Next, a more specialised introduction to the
3+1 spacetime decomposition, numerical relativity and compact objects, with focus
on boson stars, was given. After this the details of
numerical methods for PDE's and {\sc GRChombo}, the main numerical code used throughout
the thesis, were covered. Along with this, the creation of boson star initial data is given in
detail along with simple examples of their collisions in three spatial dimensions.
The head-on collision result in the prompt collapse to a black hole and the grazing
collision (a collision with small grazing parameter) results in the formation of
a black hole with a quasi-long lived scalar field configuration.

Superposing solutions in general relativity generally does not constitute a new
solution of Einstein's equation; the superposed solution will violate the Hamiltonian
and momentum constraints in Eqs(\ref{nr:eq:ham}) \, \& \, (\ref{nr:eq:mom}). While
constraint satisfying initial data for black-hole binaries is well known at this point,
the same cannot be said about boson star binaries. When superimposing
compact objects in general relativity, the overall constraint violation reduces with
increasing separation of the objects; it is common in the literature to produce
boson star binary initial data with plain superposition making the object separation
large where feasible. In chapter \ref{malaise} we have conducted symmetric boson
star collisions and compare two superposition schemes. The two superposition schemes
are the regular naive superposition of section \ref{grchombo:sec:superposition}
and the modified superposition scheme of section \ref{mal:sec:improvedsuperposition}.
Our results showed that the modified superposition scheme is a major improvement
over the naive superposition method when measuring the constraint equations. Additionally,
naive superposition was observed to induce spurious unphysical behaviour such as
excited stars and gravitational collapse. However, the modified superposition
scheme is not perfect, there is still some residual constraint violation; on the
other hand this scheme could be a useful first guess in a true constraint solver
when one becomes available. Given the simplicity of implementation and
effectiveness of the modified superposition scheme we recommend its use to the
general numerical relativity community. A future study to considering non-symmetric
binaries is certainly warranted.

\subsection{QFS stuff}

The QFS system for a generic continuity equation in curved space is derived in
section (\ref{q:qfs_system}) and is explicitly given for spherical coordinate
surfaces. The QFS system is first used to derive expressions for the Noether charge
flux density $\mathcal{F}$ of a complex scalar field and a complex Proca field;
both fields have a global $U(1)$ symmetry which gives rise to the conserved Noether charge.
Additionally, the QFS system is used to re-derive the well known Noether charge density
$\mathcal{Q}$.
Next, expressions for $\mathcal{Q}$, $\mathcal{F}$ and $\mathcal{S}$ are derived for
energy momentum-currents (of matter) where $\mathcal{S}$ represents the creation/destruction of
charge; $\mathcal{S}$ is called the source term. These expressions are then given for use
with angular momentum explicity; expressions for the energy density are also given
in agreement with the literature. The three variables give an intuitive understanding
of the behaviour of angular momentum (or any charge) where $\mathcal{Q}$ is the amount of
angular momentum in the matter field, $\mathcal{F}$ is the flux leaving an arbitrary
surface and $\mathcal{S}$ is the destruction or creation of angular momentum; the
destruction/creation of angular momentum of the matter field can be understood as
the transfer of angular momentum between matter and curvature.

In addition to the intuitive understanding of local angular momentum (and charges with
continuity equations in general) bought by the QFS system it
also has numerical uses. One use is that combining the PROPERLY CONSERVED MEASURE.
Additionally, the QFS equation can be used as an extra check on the resolution of
a numerical relativity evolution. In the continuum limit, the QFS equation [REF]
should be obeyed exactly and the error measures defined in [REF] SHOW THIS - MAYBE LINK THIS WITH TEH NUMERICAL BIT.


After the derivation and technical background of the QFS system had been derived, it
was applied to an example spacetime. This example consists of the numerical simulation
of a boson star collision using {\sc GRchombo}, described in section \ref{grchombo:sec:grchombo}.
ending in the formation of a quasi-stable object.

tested on boson star collision
maybe describe the collision
the measure was useful to quantify teh remainng angular momentum and waht is dissipated in S
agreement with newton - explain why we dont mind agreeing with newton
shows approx 3rd order convergence
erorr of $3\%$ after 8000 times units




Given the ability to measure the flux through an arbitrary surface, the QFS system is
well suited to cutting out unwanted volumes or including only desired surfaces. Including
only a given volume is useful if we want to check the resolution (or make more
accurate measures) of a region of spacetime rather than the entirety of the numerical grid. For example
in the simulation of
Cutting out pre-specified volumes could be useful in black-hole spacetimes where we
don't want to perform a measure within a certain radius of the singularity. One caveat of
this approach is that the effect of a moving volume (with respect to the coordinates)
has not been computed here and would need to be taken into account if it is to be used;
an example where we might want a moving volume is to track a black hole.

say something about movig volumes for future study?



It is hoped that the QFS system will be useful to the Numerical Relativity community
for better measurement of local energy-momentum of matter and Noether charge aswell
as powerful check on simulation resolution.





\section{Copied Stuff}



In this work a study of continuity of matter in general relativity is extended to
include angular momentum of matter and Noether currents associated with gauge
symmetries. Expressions for the Noether charge and flux of complex scalar fields
and complex Proca fields are found using this formalism. Expressions for the
angular momentum density, flux and source are also derived which are then applied
to a numerical relativity collision of boson stars in 3D with non-zero impact
parameter as an illustration of the methods.

A derivation of the QFS system (\ref{q:qfs_system}) for continuity equations,
valid locally for general spacetimes, is derived and applied to spherical integration
surfaces. Although spherical extraction surfaces are used, the methods of section
\ref{q:sect:sphere} can be applied to general extraction surfaces with minor adjustments.
The QFS system is used to calculate the well known Noether charge densities for
complex scalar and complex vector (Proca) fields along with novel expressions for
the flux variable $\mathcal{F}$ in section \ref{q:sect:noether}. Next the QFS system
for energy momentum currents associated with matter are found and the main result
of this chapter is the explicit derivation of the angular momentum QFS variables
$\mathcal{Q}$,  $\mathcal{F}$ and $\mathcal{S}$. The three variables can be used
to measure the angular momentum of matter within a region, the flux of angular
momentum of matter through the boundary of that region and the transfer of angular
momentum between matter and curvature; they can also be used with Eq.~(\ref{q:qfs_system})
to determine the numerical quality of a simulation as the QFS system is exactly
satisfied in the continuum limit. In section \ref{q:sect:results} the combination
of variables $\mathcal{Q}$ and $\mathcal{S}$ is shown to be a superior measure of
angular momentum than integrals of only the charge density $\mathcal{Q}$ in two
ways; firstly its measurement is less prone to oscillations and secondly it is
conserved in the large radius limit. THIS PARAGRAPH IS PASTED, CANT USE

The QFS system for angular momentum was then numerically tested on a dynamic non-linear
spacetime consisting of two colliding boson stars; the collision has a small impact
parameter giving rise to a non-zero total angular momentum. The stars promptly
collide and form a highly perturbed, localised scalar field configuration partially
retaining angular momentum. The total angular momentum of the spacetime is
measured using the QFS variables (Eqs.~(\ref{q:bigq}), (\ref{q:final_flux}) and
(\ref{q:s_explicit_angmom})) and is shown to agree well with the Newtonian
approximation. This is a good check on the normalisation of the QFS variables as
they should return the Newtonian calculation in the low energy limit; even though
we simulate a fully non-linear spacetime the density and boost velocity of the
stars are mild. The final numerical result is the convergence test of the QFS system
which measures the relative error described in \ref{q:sect:results}. The relative
error converges to zero with order $\omega\approx 1.9$ in the continuum limit
and the highest resolution simulation gives a fractional error of approximately
$3 \%$ in the total angular momentum after $8000$ time units. THIS PARAGRAPH IS PASTED CANT USE

\subsection{spinning bs paper stuff}

Combining the aforementioned superposition improvement scheme,



\subsection{black holes final}

Finally, the consequences of non-zero eccentricity on the gravitational wave signal
and the kick magnitude and direction
of un-equal mass Schwarzschild black-hole binaries has been studied. A kick is a
gravitational phenomenon where a system radiates linear momentum (in gravitational waves)
resuting in the centre of mass being \enquote{kicked} in the opposite direction. A series of
binaries with mass ratios $q=2/3, 1/2, {\rm and} 1/3$ which complete about 3 orbits
in the quasicircular limit are simulated. We also examine a comparison set of
simulations with $q=1/2$, and a greater initial
separation, which completes approximately six orbits in the quasicircular limit.
In all cases the black holes were initially placed on the x-axis with boosts of
equal and opposite momentum perpendicular to the x-axis; therefore the x-axis is
initially the periapsis.

In the black-hole binaries simulted, a maximum in the kick magnitude is seen for an eccentricity
of $e\sim 0.5$ and is approximately $25\%$ larger than the quasi circular case. In addition
to this, the kick magnitude is seen to oscillate, with respect to eccentricity, by
$\sim 10\%$ of the quasicircular limit. As the eccenticity tends to unity,
in the case of a headon collision, the kick magnitude becomes small. These simulations
are simulated with two codes, {\sc GRChombo} and {\sc Lean}; both codes show an error
budget of $3$-$4\,\%$.

The oscillatory behaviour of the kick magnitude is attributed to the final infall
angle (or direction) with respect to the periapsis of the black holes in merger.
The notion of an infall angle in
general relativity is not well defined and it is approximated here by considering the
angle made by the final kick to the x-axis. When the kick magnitude is plotted as a
function of infall angle (in Figure~\ref{bhkick:fig:theta-plots}) it displays the
hypothesized oscillatory behaviour with an period close to $2\pi$; any deviation
here is attributed to periapsis precession.

The dependance of the oscillatory behaviour of the kick with respect to the initial
eccenticity is explored the the aforementioned second set of simulations with
$q=1/2$ and a larger initial deparation (or a less negative binding energy). The
oscillations are more pronounced in this situation which is attributed to the longer
infall time which gives more time for phase differences between similar binaries to
build up.

While the findings of these black-hole kick simulations answer many questions, they
also ask more. For one, from a study of two sets of simulations with differnt
initial separations we are led to believe
that the oscillatory dependance of the kick on; to what extent this is hidden by
gravitational circularisation
\footnote{Gravitational circularisation refers to the natural phenomenon in general
relativity where eccentricity is damped over time eventually turning an eccentric
binary into a quasi-circular one.}
is unknown. A study of long eccentic inspirals would likely require post newtonian
methods and . Other interesting studies could include tracking the precession of
the periapsis in this strong gravity regime and the potential study of kicks in
high boost graxing collisions or even the possibility of kicks in compact objecty
binaries such as boson stars.

\subsection{stuff}
add future works to all the bits here ..
