
%\newpage
%\pagenumbering{arabic}


\section{Introduction}
\subsection{Introduction to Compact Objects and Boson Stars}
The first non flat solution to Einstein's equation found was that of the spherically symmetric, static and asymptotically flat vacuum spacetime by Karl Schwarzschild in 1915. The solution was designed to be used outside a spherically symmetric, non-spinning, body of mass; however it turned out to provide use in describing black holes. This metric was then modified by Tolman, Oppenheimer and Volkov in 1939 to describe the non-vacuum case of a constant density neutron star. This turned out to give an unphysical estimate of $0.7 M_\odot$ for the upper limit of neutron star mass due to the equation of state. 

The study of compact exotic objects can be traced back to John Wheeler who investigated Geons in 1955 for their potential similarity to elementary particles. Geons are gravito-electromagnetic objects with the name arising from "gravitational electromagnetic entity". In 1968 David Kaup published [] describing what he called "Klein-Gordon Geons", nowadays referred to as boson stars. Importantly, boson stars are a localised complex Klein-Gordon configuration, with the real counterparts being unstable. Many variants such as (Spin 1) proca stars [], electromagnetically charged boson stars and many others have been studied. 

Interest in boson stars remains for many reasons. Given the recent discovery of the higgs boson, we know that scalar fields exist in nature and any gravitational wave signals created by compact objects could theoretically be detected with modern gravitational wave interferometers. Secondly, boson stars are a good candidate for dark matter haloes. Boson stars are also useful as a proxy to other compact objects in general relativity; there is a lot of freedom in the construction of different types of boson star and they can be fine tuned to model dense neutron stars for one example. The advantage this would have over simulating a real fluid is that the Klein Gordon equation is linear in the principal part meaning smooth data must always remain smooth; thus avoiding shocks and conserving particle numbers relatively well with less sophisticated numerical schemes. 

On a slightly different topic, collisions of boson stars could be a natural method to produce scalar hair around black holes which will be discussed later in more detail. 

\subsection{Conventions}
The conventions used in this report are Geometric units, $c=G=\hbar=1$, unless stated otherwise. The metric signature will always be $(-,+,+,+)$. Finally, for quantities such as the Ricci Scalar, which differ depending on wether they are elements of the full $3+1$ dimensional manifold $\M$ or a $3$D hypersurface $\Sigma$, standard letters $(R)$ represent the object belonging to the target space and calligraphic letters $(\R)$ correspond to the projected object.




\subsection{Differential Geometry}
We all learned Pythagerous theorem at school [INSERT PICTURE OF RIGHT ANGLED TRIANGLE], this is 2d. Generalise this to algebraic metric rather than constant kroneka delta metric. Give exmaple in spherical polars. Maybe use this to calculate areas/lengths. Maybe insert picture of sphere. Insert proof that root det g is needed for volume elment from transforming a generic metric from one coord system to another? maybe need to invoke diff forms?

Differential Geometry (DG) is the extension of calculus, linear algebra and multilinear algebra to general
geometries. Einstein’s Theory of Relativity is written using the language of DG as it is the natural
way to deal with curves, tensor calculus and differential tensor equations in curved spaces. For a basic
introduction to DG, we should start with a manifold $\M$ which is an $N$ dimensional space that locally looks
like $\rspace^N$, $N$ dimensional Euclidean space. This is important as at a point $p\in\M$ we can find infinitesimally
close neighbouring points $p + \delta p \in \M$. 

A real scalar function $f$ over $\M$ maps any point $p\in \M$ to a real number, this is denoted as
$f : p \rightarrow \rspace$. An important example of a set of scalar functions is the coordinate system $\phi$, $\phi : p \rightarrow \rspace^N$,
this is normally written $x^\mu$ where $\mu\in\{0,1,...,N-1\}$ is an index labelling the coordinate. The map $\phi$ is called a chart,
and unlike Euclidean space one chart may not be enough to cover the entire manifold; in this case a set
of compatible charts should be smoothly joined, collectively known as an atlas.

Now that functions have been discussed, the next simplest object we can discuss is a curve, or path, through $\M$. A curve $\Gamma$ is a set of smoothly connected points $p(\lambda)\in \M$ that smoothly depend on an input parameter $\lambda \in [\lambda_0,\lambda_1]$. This can be expressed in terms of coordinates as $x^\mu(\lambda)$ where $\phi:p(\lambda) \rightarrow x^\mu(\lambda)$. Differentiating a function $f$ along $\Gamma$ with respect to $\lambda$ gives
\begin{equation} \label{eq:dfdl}
\frac{\dd}{\dd \lambda}f(x^\mu(\lambda)) = \frac{\dd x^\nu}{\dd \lambda}\frac{\partial f(x^\mu)}{\partial x^\nu} = \frac{\dd x^\nu}{\dd \lambda}\partial_\nu f,
\end{equation}
where $\partial_\nu = {\partial}/{\partial x^\nu}$ and the Einstein summation convention was invoked, summing over all values of $\nu$. Equation~(\ref{eq:dfdl}) was derived independantly of the choice of $f$, therefore we can generally write
\begin{equation} \label{eq:ddl}
\frac{\dd}{\dd \lambda} = \frac{\dd x^\nu}{\dd \lambda}\partial_\nu.
\end{equation}
The operator $\dd/\dd \lambda $ can act on any function $f$ and return a new function $\tilde{f}$ over $\M$, formally this is written as $\dd/\dd \lambda (f) = \tilde{f}$ where $\tilde{f}:p\in\M\rightarrow \rspace$. We can also think of $\dd/\dd \lambda$ as a vector $\bs{X}$ with components $X^\mu=\dd x^\mu / \dd \lambda$ and basis vectors $\bs{e}_\mu:=\partial_\mu$ taken from Eq.~(\ref{eq:ddl}). The vector $\bs{X}$ can be written as $\bs{X} = X^\mu \bs{e}_\mu$ and can act on a general function $f$ over $\M$ as $\bs{X}(f) = X^\mu \bs{e}_\mu(f) = X^\mu \partial_\mu f$. 

discuss coords, paths, vectoirs, forms, tensors, derivatives (lie, cov, partial ...), pullbacks (needed later), Riemann and stuff. see intro and appendix of smith kight.

\subsection{General Relativity}
Einstein vacuum eqn, then add matter, maybe horizons and black holes stuff. check harvey/tong gr/bh notes for inspiration.

\subsection{Special Relativity}
gradient of metric go to zero, drops out SR?

1887 Michelson-Morley to find aether that light moves on. Instead found not true. laws of physics are the same in all frames. Assming these two things Einstein derived the Lorentz group.

\subsection{Stuff}
wheeler quote, matter tells space how to curve, adn space tells matter how to move.

GRChombo section?

einstein summation conventions?
